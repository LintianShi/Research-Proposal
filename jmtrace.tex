\documentclass[UTF8]{article}

\usepackage{ctex}
\usepackage{amsmath}
\usepackage{amssymb}
\usepackage{hyperref}
\usepackage{geometry}
\usepackage{algorithm}  
\usepackage{algorithmicx}  
\usepackage{algpseudocode}
\usepackage{listings}
\usepackage{color}
\usepackage{xcolor}
\definecolor{dkgreen}{rgb}{0,0.6,0}
\definecolor{gray}{rgb}{0.5,0.5,0.5}
\definecolor{mauve}{rgb}{0.58,0,0.82}
\lstset{frame=tb,
     language=Java,
     aboveskip=3mm,
     belowskip=3mm,
     showstringspaces=false,
     columns=flexible,
     basicstyle = \ttfamily\small,
     numbers=none,
     numberstyle=\tiny\color{gray},
     keywordstyle=\color{blue},
     commentstyle=\color{dkgreen},
     stringstyle=\color{mauve},
     breaklines=true,
     breakatwhitespace=true,
     tabsize=4
}

\geometry{a4paper,scale=0.8}

\begin{document}

\section*{jmtrace实验报告}
MG20330050 侍林天
\section{整体框架}
jmtrace需要我们对一个java的程序在其进行内存访问操作的时候输出一些信息,因此我们需要修改一个已打包为jar包的Java程序,在其中插桩一些信息输出代码。

首先我们使用了java中的instrument包对类定义动态改变和操作,定义了一个JMTraceTransformer类在JVM加载Java程序中的类时进行动态的更改。
在JMTraceTransformer类中我们使用了org.objectweb.asm包帮助我们对jar包中的字节码进行遍历,并且感觉通过改写asm框架的函数在遍历到相应字节码时插入我们需要的字节码。
因为java中的内存读写操作都集中在Method里,所以我们主要对asm框架的MethodVisitor进行修改。

此外我们还写了一个JMTraceLogPrinter类,里面有一系列静态函数用于输出信息。

\section{代码插桩}
Java字节码中对内存进行访问的指令有getstatic/putstatic/getfield/putfield/*aload/*astore。
我们只需要考虑在遍历字节码时遇到这些指令的时候插入相应的输出代码即可。

\subsection{涉及Field的内存访问操作}
涉及到Field的内存访问操作有getstatic/putstatic/getfield/putfield。
可以使用ASM包中MethodVisitor类里的void visitFieldInsn(int opcode, String owner, String name, String descriptor)在遍历到这些操作的时候进行插桩。

\subsubsection{getstatic/putstatic}
因为getstatic和putstatic读写的是静态成员,所以我们直接通过MethodVisitor的visitLdcInsn将域的所有者owner和域名name压入栈中,
最后通过MethodVisitor的visitMethodInsn插入INVOKESTATIC调用输出函数staticGetLogPrint(staticPutLogPrint)。
这里我们直接将持有该静态成员的类名的哈希值作为对象标识。
% \begin{lstlisting}[language=Java]
%     mv.visitLdcInsn(owner);
%     mv.visitLdcInsn(name);
%     mv.visitMethodInsn(Opcodes.INVOKESTATIC, "log/JMTraceLogPrinter", "staticGetLogPrint", "(Ljava/lang/String;Ljava/lang/String;)V", false);
% \end{lstlisting}

\subsubsection{getfield}
指令getfield时operand stack如下:
\begin{equation*}
    \begin{split}
        &\ldots, objectref\ \rightarrow\\
        &\ldots, val
    \end{split}
\end{equation*}

我们需要将objectref和域名name作为参数传递给输出函数,因此首先通过MethodVisitor的visitInsn方法插入DUP指令,将栈顶的objectref复制一份,
然后再用visitLdcInsn将域名name压入栈中,最后调用静态函数输出信息。
% \begin{lstlisting}[language=Java]
%     mv.visitInsn(Opcodes.DUP);
%     mv.visitLdcInsn(name);
%     mv.visitMethodInsn(Opcodes.INVOKESTATIC, "log/JMTraceLogPrinter", "getLogPrint", "(Ljava/lang/Object;Ljava/lang/String;)V", false);
% \end{lstlisting}

\subsubsection{putfield}
指令putfield和getfield不同的地方在于,执行putfield时的栈顶多了一个要写入的值val,如左边所示。
我们无法直接在栈顶拿到object,因此我们需要做一些额外的操作,使得在调用输出函数前栈为右边那样。
\begin{equation*}
    \begin{split}
        &\text{val}\ \rightarrow\qquad\qquad\ \  \text{objectref}\ \rightarrow\\
        &\text{objectref}\qquad\qquad \text{val}\\
        &\qquad\qquad\qquad\qquad\text{objectref}
    \end{split}
\end{equation*}

如果val是一个单字节的数据如boolean, int, float等,我们只需要使用java字节码中的SWAP指令交换val和objectref,然后DUP复制栈顶的objectref作为函数参数,
函数调用完毕之后再使用SWAP将val和\\objectref换回原来的位置恢复现场即可。但如果val是一个double或者long的数据那么则会麻烦得多,
因为此时的val占了两个字节不能直接使用SWAP操作。这时需要用到DUP2\_X1将栈顶的val复制,并插入到objectref之后,
接着用POP2将栈顶的val弹出,这样我们就完成了两字节的val和object的交换。
然后我们使用DUP复制栈顶的objectref完成参数的准备。最后使用DUP\_X2将栈顶的objectref复制一份,插入两个字节后面,再POP掉栈顶的objectref,恢复现场。



\section{ACKNOWLEDGE}
https://docs.oracle.com/javase/specs/jvms/se11/html/jvms-6.html

\end{document}